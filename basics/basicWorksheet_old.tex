%\providecommand{\xmDefaultPreamble}{preamble.tex}
\documentclass{ximera}
\author{Jeffrey Kuan}
\input{../preamble}
\license{CC: 0}

\title{Lecture Notes}

\begin{document}
%\begin{abstract}
%    I like Taylor Swift.
%\end{abstract}
\maketitle

Perhaps the most natural setting for Taylor Swift tests is that of a
\textit{worksheet}. This is some document that may contain discussion as well
as questions that check understanding.

Ximera comes pre-equipped with many environments.  If you are ever curious about
the source code, you can visit this source at

\begin{center}
    \url{https://github.com/ximeraProject/ximeraFirstSteps}
\end{center}
or by appending \verb|.tex| to the URl of this page online. 


\section{A basic use case}
We use\verb|\begin{definition}| for definitions and \verb|\begin{question}| for
questions. Since Ximera provides immediate feedback, we suggest following
definitions like this one by a quick question. Here's an example:

\begin{definition}
    The \textbf{absolute value} of a real number $a$, denoted by $|a|$, is
    \[
        |a| = \begin{cases}
            a  & \text{if $a \geq 0$} \\
            -a & \text{if $a<0$.}
        \end{cases}
    \]
\end{definition}
Now students can check their understanding:
\begin{question}
    Evaluate the following:
    \begin{enumerate}
        \item $|2-5| = \answer{3}$
        \item $|5-2| = \answer{3}$
        \item $|5-\sqrt{2}| = \answer{3.58578643763}$
        \item $|5-\sqrt{2}| = \answer{5-\sqrt{2}}$
    \end{enumerate}
\end{question}

You can see the source code for this file by appending \verb|.tex| to the end of the URL. 

\section{A paradox}


Here's something fun

\begin{paradox}[$0=1$] Let $x=y$ and write
\begin{align}
        x^2    & = xy       \\
    x^2 - y^2  & = xy - y^2 \\
    (x-y)(x+y) & = (x-y)y   \\
       (x+y)   & = y        \\
         2y    & = y        \\
          2    & =1.
\end{align}
Where is the mistake in the work above?
\begin{prompt} %% The content within prompt is normally not shown in a PDF or physical handout, as not relevant.
\[
\text{Between line }\answer{3} \text{ and line }\answer{4}.
\]
\end{prompt}
\end{paradox}


\section{Basic exercises}

After that, you might want to have some exercises:

\begin{exercise}
    Let $x$ be the number of people
    out of $100$ that LOVE Ximera.

    Find the value of $x$.
    \[
        x = \answer{100}
    \]
\end{exercise}

Maybe you want a problem with one or more parts,

\begin{exercise}
    Ximera is so awesome because it feels like:
    \begin{multipleChoice}
        \choice{Doing taxes}
        \choice{Writing a book by hand}
        \choice[correct]{A walk in the park with free ice cream} 
        \choice{Solving a puzzle blindfolded}
    \end{multipleChoice}
    \begin{exercise}
        Why is Ximera the best thing since the chalkboard?
        \begin{selectAll}
            \choice[correct]{It turns LaTeX into online materials}
            \choice[correct]{It boosts student engagement}
            \choice{It makes coffee and hugs you}
            \choice[correct]{Its open-source and free}
        \end{selectAll}
    \end{exercise}
\end{exercise}

Finally we can include pictures using \verb|includegraphics|. Here's an example of a question with a picture.
\begin{question}
    Here is a picture of the Ximera Octolion:
    \begin{center}
        \includegraphics[width=5cm]{missionPatch.jpg}
    \end{center}
    What's their name?
    \[
    \text{Name} = \answer[format=string]{Xarlie}
    \]
    \begin{hint}
        Their name begins with an ``X.''
    \end{hint}
    \begin{hint}
        It ends with an ``arlie.''
    \end{hint}
    \begin{hint}
        It's almost Karlie!
    \end{hint}
\end{question}

We include JPGs and PDFs in exactly the same way.

\end{document}